\tableofcontents
\vspace*{4em}



\phantomsection
\addcontentsline{toc}{section}{EM}
\subsection*{EM}

\begin{lstlisting}
clear all
close all
%graphics_toolkit('gnuplot');
myGreen = [0 0.5 0];

a_total = load('c1.txt');
o_total = load('c2.txt');
u_total = load('c3.txt');
\end{lstlisting}


\phantomsection
\addcontentsline{toc}{section}{Separo en test y train}
\subsection*{Separo en test y train}


\begin{lstlisting}
% a
ind_perm = randperm(length(a_total));	% Entreno en cada corrida con otras muestras
a_train = a_total(ind_perm(1:35),1:2);
a_test = a_total(ind_perm(36:end),1:2);
% o
ind_perm = randperm(length(o_total));	% Entreno en cada corrida con otras muestras
o_train = o_total(ind_perm(1:35),1:2);
o_test = o_total(ind_perm(36:end),1:2);
% u
ind_perm = randperm(length(u_total));	% Entreno en cada corrida con otras muestras
u_train = u_total(ind_perm(1:35),1:2);
u_test = u_total(ind_perm(36:end),1:2);
\end{lstlisting}


\phantomsection
\addcontentsline{toc}{section}{Gráficos conjunto de puntos}
\subsection*{Gráficos conjunto de puntos}

\begin{lstlisting}
train = [a_train;o_train;u_train];

% Hago el 'scatter' de las muestras sin clasificar
figure
hold on
plot(train(:,1), train(:,2),'k.', 'MarkerSize', 10)
grid minor
close
\end{lstlisting}


\phantomsection
\addcontentsline{toc}{section}{Entrenamiento:}
\subsection*{Entrenamiento:}


\begin{lstlisting}
%Inicialización con puntos: usar 5 puntos de train con etiquetas para inicializar
%Inicialización supervisada
[a,b,c,d,e,f,g,h,i] = inicializacion(train, [3], length(a_train));
% Inicialización no supervisada (random)
%[a,b,c,d,e,f,g,h,i] = inicializacion(train, [5;'r'], length(a_train));

covg = d + e + f / 3;
\end{lstlisting}


\phantomsection
\addcontentsline{toc}{section}{Preparo iteraciones:}
\subsection*{Preparo iteraciones:}

\begin{lstlisting}
%Medias, covarianzas y pi's de las clases
ma = a;	cova = covg;	proba = g;
mo = b;	covo = covg;	probo = h;
mu = c;	covu = covg;	probu = i;

% Label
real_label = [ones(length(a_train),1);2*ones(length(o_train),1);3*ones(length(u_train),1)];

likelihood(2) = -10;
likelihood(1) = -10.1;
limlike = 0.01;
n = 2;
\end{lstlisting}


\phantomsection
\addcontentsline{toc}{section}{Preparo el gráfico}
\subsection*{Preparo el gráfico}

\begin{lstlisting}
theta = linspace(0, 2*pi, 100);
rot = [cos(theta); sin(theta)];

figure(1); hold on;
% Medias con X
plot(ma(1),ma(2), 'rx','MarkerSize',10);
plot(mo(1),mo(2),'x','color',myGreen,'MarkerSize',10);
plot(mu(1),mu(2), 'bx','MarkerSize',10);
% Puntos bien clasificados
plot(train(:,1).*(real_label==1), train(:,2).*(real_label==1), 'r.', 'MarkerSize',10);
plot(train(:,1).*(real_label==2), train(:,2).*(real_label==2), '.', 'color', myGreen, 'MarkerSize',10);
plot(train(:,1).*(real_label==3), train(:,2).*(real_label==3), 'b.', 'MarkerSize',10);
% ELIPSES
elipsea = (chol(cova)' * rot)' + ma;
elipseo = (chol(covo)' * rot)' + mo;
elipseu = (chol(covu)' * rot)' + mu;
plot(elipsea(:,1),elipsea(:,2),'r');
plot(elipseo(:,1),elipseo(:,2),'color', myGreen);
plot(elipseu(:,1),elipseu(:,2),'b');
axis tight

titles = 'Iteracion ';
title([titles, num2str(0)]);
\end{lstlisting}


\phantomsection
\addcontentsline{toc}{section}{Iteración principal}
\subsection*{Iteración principal}

\begin{lstlisting}
bool_plot = 1;		% Si quiero ver los grafs de la iteración => 1. Sino 0.
N = length(train);

while( abs((likelihood(n-1) - likelihood(n))) > limlike && n<20)
	% %%% PASO E %%% %
	% Calculo de responsabilidades para cada muestra
	for i = 1:length(train)
		x = train(i,:);
		res(1) = mvnpdf(x, ma, cova)*proba;
		res(2) = mvnpdf(x, mo, covo)*probo;
		res(3) = mvnpdf(x, mu, covu)*probu;
		den = sum(res);

		gama(i,:) = res/den;
	end

	% %%% PASO M %%% %
	NA = sum(gama(:,1));	NO = sum(gama(:,2));	NU = sum(gama(:,3));
	% Recalculo:
	% media
	ma = sum(train.*gama(:,1))/sum(gama(:,1));
	mo = sum(train.*gama(:,2))/sum(gama(:,2));
	mu = sum(train.*gama(:,3))/sum(gama(:,3));

	% covarianza
	cova = estim_cov(train, ma, gama(:,1));
	covo = estim_cov(train, mo, gama(:,2));
	covu = estim_cov(train, mu, gama(:,3));

	% pi (probabilidad)
	proba = sum(gama(:,1))/sum(sum(gama));
	probo = NO/N;
	probu = NU/N;

	% %%% LIKELIHOOD %%% %
	den = sum(mvnpdf(train,ma,cova)*proba) + sum(mvnpdf(train,mo,covo)*probo) + sum(mvnpdf(train,mu,covu)*probu);

	%log_likelihood
	likelihood(n+1) = sum(log(den));

	if(bool_plot)
		% Gráfico
		figure(n); hold on;
		% Medias con X
		plot(mu(1),mu(2), 'bx','MarkerSize',10);
		plot(ma(1),ma(2), 'rx','MarkerSize',10);
		plot(mo(1),mo(2),'x','color',myGreen,'MarkerSize',10);
		% Puntos bien clasificados
		plot(train(:,1).*(real_label==3), train(:,2).*(real_label==3), 'b.', 'MarkerSize',10);
		plot(train(:,1).*(real_label==1), train(:,2).*(real_label==1), 'r.', 'MarkerSize',10);
		plot(train(:,1).*(real_label==2), train(:,2).*(real_label==2), '.', 'color', myGreen, 'MarkerSize',10);
		% Puntos clasificados
		scatter(train(:,1), train(:,2), 10, gama);
		% ELIPSES
		elipsea = (chol(cova)' * rot)' + ma;
		elipseo = (chol(covo)' * rot)' + mo;
		elipseu = (chol(covu)' * rot)' + mu;
		plot(elipseu(:,1),elipseu(:,2),'b');
		plot(elipsea(:,1),elipsea(:,2),'r');
		plot(elipseo(:,1),elipseo(:,2),'color', myGreen);

		xlabel('Primer formante [Hz]');
		ylabel('Segundo formante [Hz]');
		axis tight
		title([titles, num2str(n-1)]);
	end

	n += 1;
end
\end{lstlisting}

\graficarPNG{graf_Figura1}{Antes de la iteración.}{fig:1}
\graficarPNG{graf_Figura2}{Tras la primer iteración.}{fig:2}
\graficarPNG{graf_Figura3}{Tras la segunda iteación.}{fig:3}
\graficarPNG{graf_Figura4}{Tras la tercer iteración.}{fig:4}
\graficarPNG{graf_Figura5}{Tras la cuarta iteración.}{fig:5}
\graficarPNG{graf_Figura6}{Tras la quinta iteación.}{fig:6}
\graficarPNG{graf_Figura7}{Tras la sexta y última iteración.}{fig:7}

\phantomsection
\addcontentsline{toc}{section}{Plots de datos tras entrenamiento}
\subsection*{Plots de datos tras entrenamiento}



Es el último gráfico de la iteración



\phantomsection
\addcontentsline{toc}{section}{Plot likelihood}
\subsection*{Plot likelihood}

\begin{lstlisting}
figure
plot(likelihood);
title('Logaritmo de la verosimilitud en función de las iteraciones');
xlabel('Iteración');
ylabel('log(likelihood)');
\end{lstlisting}

\graficarPNG{graf_likelihood}{Verosimilitud en función de las iteraciones.}{fig:7}

Notesé que hubieron 6 iteraciones y en el gráfico aparecen 8. Ésto se debe a que para iniciar la iteración se tuvo que inicializar la verosimilitud para 2 valores previos que no corresponden a valores obtenidos de la verosimilitud.


\phantomsection
\addcontentsline{toc}{section}{Test}
\subsection*{Test}

\begin{lstlisting}
test = [a_test;o_test;u_test];
test_real_label = [ones(length(a_test),1);2*ones(length(o_test),1);3*ones(length(u_test),1)];

for k=1:length(test)
		disc(k,1) = discriminante(test(k,:), ma, cova, proba);
		disc(k,2) = discriminante(test(k,:), mo, covo, probo);
		disc(k,3) = discriminante(test(k,:), mu, covu, probu);
		[a,b] = max(disc(k,:));
		test_label(k) = b;
end

% Gráfico
figure()
hold on

% Medias
plot(ma(1),ma(2), 'rx','MarkerSize',20);
plot(mo(1),mo(2),'x','color',myGreen,'MarkerSize',20);
plot(mu(1),mu(2), 'bx','MarkerSize',20);

% Puntos encontrados
plot(test(:,1).*(test_label==1)', test(:,2).*(test_label==1)', 'ro');
plot(test(:,1).*(test_label==2)', test(:,2).*(test_label==2)', 'o', 'color', myGreen);
plot(test(:,1).*(test_label==3)', test(:,2).*(test_label==3)', 'bo');

% Puntos reales
plot(test(:,1).*(test_real_label==1), test(:,2).*(test_real_label==1), 'rx');
plot(test(:,1).*(test_real_label==2), test(:,2).*(test_real_label==2), 'x', 'color', myGreen);
plot(test(:,1).*(test_real_label==3), test(:,2).*(test_real_label==3), 'bx');




ErrorRatio = sum(test_label'==test_real_label)*100/length(test_label)

title(['Test con ErrorRatio de ' num2str(ErrorRatio,4) '%'])
\end{lstlisting}
\begin{lstlisting}[language={},xleftmargin=5pt,frame=none]
ErrorRatio = 95.556 

\end{lstlisting}

\graficarPNG{graf_test}{Puntos de test clasificados con un \num{95.56}\% de precisión.}{fig:9}

